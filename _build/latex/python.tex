%% Generated by Sphinx.
\def\sphinxdocclass{report}
\documentclass[letterpaper,10pt,english]{sphinxmanual}
\ifdefined\pdfpxdimen
   \let\sphinxpxdimen\pdfpxdimen\else\newdimen\sphinxpxdimen
\fi \sphinxpxdimen=.75bp\relax

\PassOptionsToPackage{warn}{textcomp}
\usepackage[utf8]{inputenc}
\ifdefined\DeclareUnicodeCharacter
% support both utf8 and utf8x syntaxes
  \ifdefined\DeclareUnicodeCharacterAsOptional
    \def\sphinxDUC#1{\DeclareUnicodeCharacter{"#1}}
  \else
    \let\sphinxDUC\DeclareUnicodeCharacter
  \fi
  \sphinxDUC{00A0}{\nobreakspace}
  \sphinxDUC{2500}{\sphinxunichar{2500}}
  \sphinxDUC{2502}{\sphinxunichar{2502}}
  \sphinxDUC{2514}{\sphinxunichar{2514}}
  \sphinxDUC{251C}{\sphinxunichar{251C}}
  \sphinxDUC{2572}{\textbackslash}
\fi
\usepackage{cmap}
\usepackage[T1]{fontenc}
\usepackage{amsmath,amssymb,amstext}
\usepackage{babel}



\usepackage{times}
\expandafter\ifx\csname T@LGR\endcsname\relax
\else
% LGR was declared as font encoding
  \substitutefont{LGR}{\rmdefault}{cmr}
  \substitutefont{LGR}{\sfdefault}{cmss}
  \substitutefont{LGR}{\ttdefault}{cmtt}
\fi
\expandafter\ifx\csname T@X2\endcsname\relax
  \expandafter\ifx\csname T@T2A\endcsname\relax
  \else
  % T2A was declared as font encoding
    \substitutefont{T2A}{\rmdefault}{cmr}
    \substitutefont{T2A}{\sfdefault}{cmss}
    \substitutefont{T2A}{\ttdefault}{cmtt}
  \fi
\else
% X2 was declared as font encoding
  \substitutefont{X2}{\rmdefault}{cmr}
  \substitutefont{X2}{\sfdefault}{cmss}
  \substitutefont{X2}{\ttdefault}{cmtt}
\fi


\usepackage[Bjarne]{fncychap}
\usepackage[,numfigreset=1,mathnumfig]{sphinx}

\fvset{fontsize=\small}
\usepackage{geometry}


% Include hyperref last.
\usepackage{hyperref}
% Fix anchor placement for figures with captions.
\usepackage{hypcap}% it must be loaded after hyperref.
% Set up styles of URL: it should be placed after hyperref.
\urlstyle{same}
\addto\captionsenglish{\renewcommand{\contentsname}{Coding in GIS I}}

\usepackage{sphinxmessages}




\title{Coding in GIS}
\date{Sep 01, 2020}
\release{}
\author{Nils Ratnaweera}
\newcommand{\sphinxlogo}{\vbox{}}
\renewcommand{\releasename}{}
\makeindex
\begin{document}

\pagestyle{empty}
\sphinxmaketitle
\pagestyle{plain}
\sphinxtableofcontents
\pagestyle{normal}
\phantomsection\label{\detokenize{intro::doc}}


Dieses Dokument ist Teil des ZHAW AGI Modul und ist inhalt der drei Blöcke “Coding in GIS”.

Noch ein kurzer Hinweis zur Typografie dieses Dokumentes:
\begin{itemize}
\item {} 
Wenn sich im Fliesstext (Python oder ) Code befindet, wird \sphinxcode{\sphinxupquote{dieser Festschriftart}} dargestellt

\item {} 
Alleinstehende Codezeilen werden  !! HIER NOCH BESCHREIBEN!!!

\item {} 
Englische Fachbegriffe, deren Übersetzung eher verwirrend als nützlich sein würde, werden \sphinxstyleemphasis{kursiv} hervorgehoben

\item {} 
Für R\sphinxhyphen{}Nutzer stelle ich den Vergleich zur R\sphinxhyphen{}Programmiersprache her. Wer wenig Erfahrung in R hat kann diese Teile ignorieren

\end{itemize}


\chapter{Python Basics}
\label{\detokenize{01_02_Python_Basics:python-basics}}\label{\detokenize{01_02_Python_Basics::doc}}

\section{Übung 1: Variablen erstellen}
\label{\detokenize{01_02_Python_Basics:ubung-1-variablen-erstellen}}
Wir beginnen damit, einfache Datentypen wie Zahlen, Text, boolsche Variabeln (siehe Vorlesungsfolien) sogenannten Variablen zu zuweisen. Führe die folgenden Schritte aus und schau dir nach jedem Schritt die erstellten Variablen jeweils im im Variable explorer von Spyder an
\begin{enumerate}
\sphinxsetlistlabels{\arabic}{enumi}{enumii}{}{.}%
\item {} 
Erstelle eine Variabel vorname mit deinem Vornamen

\item {} 
Erstelle eine zweite Variabel nachname mit deinem Nachnamen

\item {} 
Was sind vorname und nachname für Datentypen?

\item {} 
Klebe die beiden Variabeln zusammen, ohne den Output zu speichern

\item {} 
Erstelle eine Variabel groesse mit deiner Körpergrösse in Zentimeter auf Zentimeter genau. Was ist das für ein Datentyp?

\item {} 
Ermittle deine Grösse in Meter auf der Basis von groesse. Was ist das für ein Datentyp?

\item {} 
Erstelle eine boolsche Variable blond und setzte sie auf True wenn diese Eigenschaft auf dich zutrifft und False falls nicht.

\end{enumerate}

\sphinxstylestrong{Für R Nutzer}:
\begin{itemize}
\item {} 
In R gibt es die gleichen Primitiven Datentypen wie in Python: Boolean, String, Integer, Float. Die Bezeichnung ist jedoch leicht anders (Logical, Character, Integer, Numeric, Double)

\item {} 
In R können Variablen genau gleich zugewiesen werden wie in Python (mit dem \sphinxcode{\sphinxupquote{=}} Symbol). In R gibt es zudem noch die Möglichkeit, Variablen mit \sphinxcode{\sphinxupquote{\textless{}\sphinxhyphen{}}} zu zuweisen, in Python jedoch nicht

\end{itemize}


\section{Übung 2: Lists}
\label{\detokenize{01_02_Python_Basics:ubung-2-lists}}\begin{enumerate}
\sphinxsetlistlabels{\arabic}{enumi}{enumii}{}{.}%
\item {} 
Erstelle eine Variable \sphinxcode{\sphinxupquote{vornamen}} bestehend aus einer \sphinxstyleemphasis{List} mit 3 Vornamen

\item {} 
Erstelle eine zweite Variable \sphinxcode{\sphinxupquote{nachnamen}} bestehend aus einer \sphinxstyleemphasis{List} mit 3 Nachnamen

\item {} 
Erstelle eine Variable \sphinxcode{\sphinxupquote{groessen}} bestehend aus einer \sphinxstyleemphasis{List} mit 3 Grössenangaben in Zentimeter.

\end{enumerate}

\sphinxstylestrong{Für R Nutzer}: Eine Python List entspricht eines R Vectors

\begin{sphinxVerbatim}[commandchars=\\\{\}]
\PYG{n}{vornamen} \PYG{o}{=} \PYG{p}{[}\PYG{l+s+s2}{\PYGZdq{}}\PYG{l+s+s2}{Christopher}\PYG{l+s+s2}{\PYGZdq{}}\PYG{p}{,} \PYG{l+s+s2}{\PYGZdq{}}\PYG{l+s+s2}{Henning}\PYG{l+s+s2}{\PYGZdq{}}\PYG{p}{,} \PYG{l+s+s2}{\PYGZdq{}}\PYG{l+s+s2}{Severin}\PYG{l+s+s2}{\PYGZdq{}}\PYG{p}{]}
\PYG{n}{nachnamen} \PYG{o}{=} \PYG{p}{[}\PYG{l+s+s2}{\PYGZdq{}}\PYG{l+s+s2}{Annen}\PYG{l+s+s2}{\PYGZdq{}}\PYG{p}{,}\PYG{l+s+s2}{\PYGZdq{}}\PYG{l+s+s2}{May}\PYG{l+s+s2}{\PYGZdq{}}\PYG{p}{,} \PYG{l+s+s2}{\PYGZdq{}}\PYG{l+s+s2}{Kantereit}\PYG{l+s+s2}{\PYGZdq{}}\PYG{p}{]}

\PYG{n}{groessen} \PYG{o}{=} \PYG{p}{[}\PYG{l+m+mi}{174}\PYG{p}{,} \PYG{l+m+mi}{182}\PYG{p}{,} \PYG{l+m+mi}{162}\PYG{p}{]}
\end{sphinxVerbatim}


\section{Übung 3: Elemente aus Liste ansprechen}
\label{\detokenize{01_02_Python_Basics:ubung-3-elemente-aus-liste-ansprechen}}
Wie erhältst du den ersten Eintrag in der Variable \sphinxcode{\sphinxupquote{vornamen}}? Wie erhältst du den letzten Eintrag? Tipp: nutze dazu \sphinxcode{\sphinxupquote{{[}}} und \sphinxcode{\sphinxupquote{{]}}} sowie eine Zahl.

\sphinxstylestrong{Für R Nutzer}: In R werden Elemente aus Vectors auf die gleiche Weise extrahiert. Nur ist in R das erste Element die Nummer 1, Python beginnt bei 0


\section{Übung 4: Liste ergänzen}
\label{\detokenize{01_02_Python_Basics:ubung-4-liste-erganzen}}
Listen können durch der Method \sphinxcode{\sphinxupquote{append}} ergänzt werden.

\begin{sphinxVerbatim}[commandchars=\\\{\}]
\PYG{n}{vornamen}\PYG{o}{.}\PYG{n}{append}\PYG{p}{(}\PYG{l+s+s2}{\PYGZdq{}}\PYG{l+s+s2}{Malte}\PYG{l+s+s2}{\PYGZdq{}}\PYG{p}{)}
\end{sphinxVerbatim}

Ergänze in die Listen \sphinxcode{\sphinxupquote{vornamen}}, \sphinxcode{\sphinxupquote{nachnamen}} und \sphinxcode{\sphinxupquote{groessen}} durch je einen Eintrag.

\begin{sphinxVerbatim}[commandchars=\\\{\}]
\PYG{n}{nachnamen}\PYG{o}{.}\PYG{n}{append}\PYG{p}{(}\PYG{l+s+s2}{\PYGZdq{}}\PYG{l+s+s2}{Huck}\PYG{l+s+s2}{\PYGZdq{}}\PYG{p}{)}

\PYG{n}{groessen}\PYG{o}{.}\PYG{n}{append}\PYG{p}{(}\PYG{l+m+mi}{177}\PYG{p}{)}
\end{sphinxVerbatim}


\section{Input: Dictionary (Teil I)}
\label{\detokenize{01_02_Python_Basics:input-dictionary-teil-i}}
Ähnlich wie eine List, ist eine Dictionary ein Behälter wo mehrere Elemente abgespeichert werden können. Wie bei einem Wörterbuch bekommt jedes Element ein “Schlüsselwort”, mit dem man den Eintrag finden kann.
Unter dem Eintrag “trump” findet man im Langenscheidt Wörterbuch (1977) die Erklärung “erdichten, schwindeln, sich aus den Fingern saugen”.

\sphinxincludegraphics{{trump}.jpg}

In Python würde man diese \sphinxstyleemphasis{Dictionary} folgendermassen erstellen:

\begin{sphinxVerbatim}[commandchars=\\\{\}]
\PYG{n}{langenscheidt} \PYG{o}{=} \PYG{p}{\PYGZob{}}\PYG{l+s+s2}{\PYGZdq{}}\PYG{l+s+s2}{trump}\PYG{l+s+s2}{\PYGZdq{}}\PYG{p}{:}\PYG{l+s+s2}{\PYGZdq{}}\PYG{l+s+s2}{erdichten\PYGZhy{} schwindeln\PYGZhy{} sich aus den Fingern saugen}\PYG{l+s+s2}{\PYGZdq{}}\PYG{p}{\PYGZcb{}}
\end{sphinxVerbatim}

Schlüssel (von nun an mit \sphinxstyleemphasis{Key} bezeichnet) des Eintrages lautet “trump” und der dazugehörige Wert (\sphinxstyleemphasis{Value}) “erdichten\sphinxhyphen{} schwindeln\sphinxhyphen{} aus den Fingern saugen”. Beachte die geschweiften Klammern (\sphinxcode{\sphinxupquote{\{}} und \sphinxcode{\sphinxupquote{\}}}) bei der Erstellung einer Dictionary.

Eine \sphinxstyleemphasis{Dictionary} besteht aber meistens nicht aus einem, sondern aus mehreren Einträgen: Diese werden Kommagetrennt aufgeführt.

\begin{sphinxVerbatim}[commandchars=\\\{\}]
\PYG{n}{langenscheidt} \PYG{o}{=} \PYG{p}{\PYGZob{}}\PYG{l+s+s2}{\PYGZdq{}}\PYG{l+s+s2}{trump}\PYG{l+s+s2}{\PYGZdq{}}\PYG{p}{:}\PYG{l+s+s2}{\PYGZdq{}}\PYG{l+s+s2}{erdichten\PYGZhy{} schwindeln\PYGZhy{} sich aus den Fingern saugen}\PYG{l+s+s2}{\PYGZdq{}}\PYG{p}{,} \PYG{l+s+s2}{\PYGZdq{}}\PYG{l+s+s2}{trumpery}\PYG{l+s+s2}{\PYGZdq{}}\PYG{p}{:}\PYG{l+s+s2}{\PYGZdq{}}\PYG{l+s+s2}{Plunder\PYGZhy{} Ramsch\PYGZhy{} Schund}\PYG{l+s+s2}{\PYGZdq{}}\PYG{p}{\PYGZcb{}}
\end{sphinxVerbatim}

Der Clou der \sphinxstyleemphasis{Dictionary} ist, dass man nun einen Eintrag mittels dem \sphinxstyleemphasis{Key} aufrufen kann. Wenn wir also nun wissen wollen was “trump” heisst, ermitteln wir dies mit der nachstehenden Codezeile:

\begin{sphinxVerbatim}[commandchars=\\\{\}]
\PYG{n}{langenscheidt}\PYG{p}{[}\PYG{l+s+s2}{\PYGZdq{}}\PYG{l+s+s2}{trump}\PYG{l+s+s2}{\PYGZdq{}}\PYG{p}{]}
\end{sphinxVerbatim}

\begin{sphinxVerbatim}[commandchars=\\\{\}]
\PYGZsq{}erdichten\PYGZhy{} schwindeln\PYGZhy{} sich aus den Fingern saugen\PYGZsq{}
\end{sphinxVerbatim}

\sphinxstylestrong{Für R Nutzer}: Eine Python \sphinxstyleemphasis{Dictionary} entspricht einer R \sphinxstyleemphasis{List}. Diese werden in R nicht gleich erstellt, dafür aber auf die gleiche weise abgefragt wie \sphinxstyleemphasis{Dictionaries} in Python


\section{Übung 5: Dictionary}
\label{\detokenize{01_02_Python_Basics:ubung-5-dictionary}}
Erstelle eine \sphinxstyleemphasis{Dictionary} mit folgenden Einträgen: Vorname und Nachname von (d)einer Person. Weise diese Dictionary der Variable \sphinxcode{\sphinxupquote{me}} zu.

\begin{sphinxVerbatim}[commandchars=\\\{\}]
\PYG{n}{me} \PYG{o}{=} \PYG{p}{\PYGZob{}}\PYG{l+s+s2}{\PYGZdq{}}\PYG{l+s+s2}{vorname}\PYG{l+s+s2}{\PYGZdq{}}\PYG{p}{:} \PYG{l+s+s2}{\PYGZdq{}}\PYG{l+s+s2}{Charles}\PYG{l+s+s2}{\PYGZdq{}}\PYG{p}{,} \PYG{l+s+s2}{\PYGZdq{}}\PYG{l+s+s2}{nachname}\PYG{l+s+s2}{\PYGZdq{}}\PYG{p}{:} \PYG{l+s+s2}{\PYGZdq{}}\PYG{l+s+s2}{Darwin}\PYG{l+s+s2}{\PYGZdq{}}\PYG{p}{\PYGZcb{}}
\end{sphinxVerbatim}


\section{Übung 6: Elemente aus Dictionary ansprechen}
\label{\detokenize{01_02_Python_Basics:ubung-6-elemente-aus-dictionary-ansprechen}}
Rufe verschiedene Elemente aus der Dictionary via dem \sphinxstyleemphasis{Key} ab.

\begin{sphinxVerbatim}[commandchars=\\\{\}]
\PYG{n}{me}\PYG{p}{[}\PYG{l+s+s2}{\PYGZdq{}}\PYG{l+s+s2}{vorname}\PYG{l+s+s2}{\PYGZdq{}}\PYG{p}{]}

\PYG{n}{me}\PYG{p}{[}\PYG{l+s+s2}{\PYGZdq{}}\PYG{l+s+s2}{nachname}\PYG{l+s+s2}{\PYGZdq{}}\PYG{p}{]}
\end{sphinxVerbatim}

\begin{sphinxVerbatim}[commandchars=\\\{\}]
\PYGZsq{}Darwin\PYGZsq{}
\end{sphinxVerbatim}


\section{Übung 7: Key ergänzen}
\label{\detokenize{01_02_Python_Basics:ubung-7-key-erganzen}}
Um einer \sphinxstyleemphasis{Dictionary} mit einem weiteren Eintrag zu ergänzen, geht man sehr ähnlich vor wie beim Abrufen von Einträgen.

\begin{sphinxVerbatim}[commandchars=\\\{\}]
\PYG{n}{langenscheidt}\PYG{p}{[}\PYG{l+s+s2}{\PYGZdq{}}\PYG{l+s+s2}{trumpet}\PYG{l+s+s2}{\PYGZdq{}}\PYG{p}{]} \PYG{o}{=} \PYG{l+s+s2}{\PYGZdq{}}\PYG{l+s+s2}{trompete}\PYG{l+s+s2}{\PYGZdq{}} 
\end{sphinxVerbatim}

Ergänze gemäss nachstehendem Beispiel die Variable \sphinxcode{\sphinxupquote{me}} durch den Eintrag \sphinxcode{\sphinxupquote{groesse}}.

\sphinxstylestrong{Für R Nutzer}: Python \sphinxstyleemphasis{Dictionaries} werden nicht nur gleich abgerufen wie R \sphinxstyleemphasis{Lists}, sonder auch gleich ergänzt.

\begin{sphinxVerbatim}[commandchars=\\\{\}]
\PYG{n}{me}\PYG{p}{[}\PYG{l+s+s2}{\PYGZdq{}}\PYG{l+s+s2}{groesse}\PYG{l+s+s2}{\PYGZdq{}}\PYG{p}{]} \PYG{o}{=} \PYG{l+m+mi}{181}
\end{sphinxVerbatim}


\section{Input: Dictionary (Teil II)}
\label{\detokenize{01_02_Python_Basics:input-dictionary-teil-ii}}
Ein \sphinxstyleemphasis{Key} kann auch mehrere Einträge enthalten. An unserem Langenscheidts Beispiel: Das Wort “trump” ist zwar eindeutig, doch “trumpery” hat vier verschiedene Bedeutungen. In so einem Fall können wir einem Eintrag auch eine \sphinxstyleemphasis{List} von Werten zuweisen. Beachte die Eckigen Klammern und die Kommas, welche die Listeneinträge voneinander trennt.

\begin{sphinxVerbatim}[commandchars=\\\{\}]
\PYG{n}{langenscheidt}\PYG{p}{[}\PYG{l+s+s2}{\PYGZdq{}}\PYG{l+s+s2}{trumpery}\PYG{l+s+s2}{\PYGZdq{}}\PYG{p}{]} \PYG{o}{=} \PYG{p}{[}\PYG{l+s+s2}{\PYGZdq{}}\PYG{l+s+s2}{Plunder\PYGZhy{} Ramsch\PYGZhy{} Schund}\PYG{l+s+s2}{\PYGZdq{}}\PYG{p}{,} \PYG{l+s+s2}{\PYGZdq{}}\PYG{l+s+s2}{Gewäsch\PYGZhy{} Quatsch}\PYG{l+s+s2}{\PYGZdq{}}\PYG{p}{,} \PYG{l+s+s2}{\PYGZdq{}}\PYG{l+s+s2}{Schund\PYGZhy{} Kitsch}\PYG{l+s+s2}{\PYGZdq{}}\PYG{p}{,} \PYG{l+s+s2}{\PYGZdq{}}\PYG{l+s+s2}{billig\PYGZhy{} nichtssagend}\PYG{l+s+s2}{\PYGZdq{}}\PYG{p}{]}    
\PYG{n}{langenscheidt}\PYG{p}{[}\PYG{l+s+s2}{\PYGZdq{}}\PYG{l+s+s2}{trumpery}\PYG{l+s+s2}{\PYGZdq{}}\PYG{p}{]}
\end{sphinxVerbatim}

\begin{sphinxVerbatim}[commandchars=\\\{\}]
[\PYGZsq{}Plunder\PYGZhy{} Ramsch\PYGZhy{} Schund\PYGZsq{},
 \PYGZsq{}Gewäsch\PYGZhy{} Quatsch\PYGZsq{},
 \PYGZsq{}Schund\PYGZhy{} Kitsch\PYGZsq{},
 \PYGZsq{}billig\PYGZhy{} nichtssagend\PYGZsq{}]
\end{sphinxVerbatim}


\section{Übung 8: Dictionary mit List}
\label{\detokenize{01_02_Python_Basics:ubung-8-dictionary-mit-list}}
Erstelle eine neue Dictionary mit den gleichen Keys wie \sphinxcode{\sphinxupquote{me}}, aber diesmal mit mehreren Einträgen pro \sphinxstyleemphasis{Key} (also mehreren Vornamen, Nachnamen usw.). Beachte, dass nun jeder Eintrag eine \sphinxstyleemphasis{List} sein muss. Weise diese Dictionary der Variabel \sphinxcode{\sphinxupquote{people}} zu.

\begin{sphinxVerbatim}[commandchars=\\\{\}]
\PYG{n}{people} \PYG{o}{=} \PYG{p}{\PYGZob{}}\PYG{l+s+s2}{\PYGZdq{}}\PYG{l+s+s2}{vornamen}\PYG{l+s+s2}{\PYGZdq{}}\PYG{p}{:} \PYG{p}{[}\PYG{l+s+s2}{\PYGZdq{}}\PYG{l+s+s2}{Christopher}\PYG{l+s+s2}{\PYGZdq{}}\PYG{p}{,} \PYG{l+s+s2}{\PYGZdq{}}\PYG{l+s+s2}{Henning}\PYG{l+s+s2}{\PYGZdq{}}\PYG{p}{,} \PYG{l+s+s2}{\PYGZdq{}}\PYG{l+s+s2}{Severin}\PYG{l+s+s2}{\PYGZdq{}}\PYG{p}{]}\PYG{p}{,} \PYG{l+s+s2}{\PYGZdq{}}\PYG{l+s+s2}{nachnamen}\PYG{l+s+s2}{\PYGZdq{}}\PYG{p}{:} \PYG{p}{[}\PYG{l+s+s2}{\PYGZdq{}}\PYG{l+s+s2}{Annen}\PYG{l+s+s2}{\PYGZdq{}}\PYG{p}{,}\PYG{l+s+s2}{\PYGZdq{}}\PYG{l+s+s2}{May}\PYG{l+s+s2}{\PYGZdq{}}\PYG{p}{,} \PYG{l+s+s2}{\PYGZdq{}}\PYG{l+s+s2}{Kantereit}\PYG{l+s+s2}{\PYGZdq{}}\PYG{p}{]}\PYG{p}{,} \PYG{l+s+s2}{\PYGZdq{}}\PYG{l+s+s2}{groessen}\PYG{l+s+s2}{\PYGZdq{}}\PYG{p}{:} \PYG{p}{[}\PYG{l+m+mi}{174}\PYG{p}{,} \PYG{l+m+mi}{182}\PYG{p}{,} \PYG{l+m+mi}{162}\PYG{p}{]}\PYG{p}{\PYGZcb{}}
\end{sphinxVerbatim}


\section{Übung 9: Functions}
\label{\detokenize{01_02_Python_Basics:ubung-9-functions}}
Bisher haben wir Objekte erstellt. Richtig interessant wird programmieren aber erst, wenn wir mit Objekte durch Funktionen (\sphinxstyleemphasis{Functions}) verändern. \sphinxstyleemphasis{Functions} führen bestimmte Tasks aus. Wende die Functions \sphinxcode{\sphinxupquote{len()}} und \sphinxcode{\sphinxupquote{sum()}} an den Listen \sphinxcode{\sphinxupquote{groessen}} und \sphinxcode{\sphinxupquote{vornamen}} an.

\begin{sphinxVerbatim}[commandchars=\\\{\}]
\PYG{n+nb}{len}\PYG{p}{(}\PYG{n}{groessen}\PYG{p}{)}
\end{sphinxVerbatim}

\begin{sphinxVerbatim}[commandchars=\\\{\}]
4
\end{sphinxVerbatim}

\begin{sphinxVerbatim}[commandchars=\\\{\}]
\PYG{n+nb}{sum}\PYG{p}{(}\PYG{n}{groessen}\PYG{p}{)}
\end{sphinxVerbatim}

\begin{sphinxVerbatim}[commandchars=\\\{\}]
695
\end{sphinxVerbatim}

\begin{sphinxVerbatim}[commandchars=\\\{\}]
\PYG{n+nb}{len}\PYG{p}{(}\PYG{n}{vornamen}\PYG{p}{)}
\end{sphinxVerbatim}

\begin{sphinxVerbatim}[commandchars=\\\{\}]
4
\end{sphinxVerbatim}

\begin{sphinxVerbatim}[commandchars=\\\{\}]
\PYG{c+c1}{\PYGZsh{} sum(vornamen)}
\end{sphinxVerbatim}


\chapter{Python Modules}
\label{\detokenize{01_03_Python_Module:python-modules}}\label{\detokenize{01_03_Python_Module::doc}}
Ähnlich wie R basiert Python auf Erweiterungen: Diese Erweiterungen heissen in R \sphinxstyleemphasis{Libraries} / \sphinxstyleemphasis{Packages}, in Python werden sie \sphinxstyleemphasis{Modules} genannt. Sie sind dazu da, gewisse Teilbereiche unseres Arbeitsprozesses zu vereinfachen. Eine Analogie dazu: Um ein Haus zu bauen sind wir auf verschiedene Spezialisten / Spezialistinnen angewiesen: Wir brauchen zum Beispiel eine Malerin oder einen Maler. Im Telfonbuch sind seitenweise Maler*innen aufgelistet, und jede*r arbeitet etwas anders. Um eine spezifische Malerin anzuheuern müssen wir zuerst den Kontakt herstellen und die Vertragsmodalitäten vereinbaren. Erst dann können wir sie in unseren Arbeitsprozess (Haus bauen) einbinden.

Um diese Analogie auf unser Projekt zu übertragen: Das “Haus bauen” ist unser Forschungsprojekt (z.B. eine Bachelorarbeit). Ein “Telefonbuch”, wo die Spezialisten erfasst sind nennt ein \sphinxstyleemphasis{Reposoritory}. Den Erstkontakt mit der Malerin zu erstellen und die Vertragsmodalitäten zu vereinbaren bedeutet, die Erweiterung zu installieren. In R wird eine Erweiterung folgendermassen installiert:

\begin{sphinxVerbatim}[commandchars=\\\{\}]
\PYG{n}{install}\PYG{o}{.}\PYG{n}{packages}\PYG{p}{(}\PYG{l+s+s2}{\PYGZdq{}}\PYG{l+s+s2}{malerinMina}\PYG{l+s+s2}{\PYGZdq{}}\PYG{p}{)}
\end{sphinxVerbatim}

In diesem Beispiel heisst die Erweiterung \sphinxcode{\sphinxupquote{malerinMina}}. Das \sphinxstyleemphasis{Reposoritory} geben wir in R oft nicht an, weil in RStudio typischerweise schon eine Adresse hinterlegt ist. Wie wir in Python ein Modul installieren lernen wir später. Nehmen wir an dieser Stelle an, wir haben die Erweiterung \sphinxcode{\sphinxupquote{malerinMina}} in Python installiert. Treiben wir an dieser Stelle die Analogie etwas weiter: Der Erstkontakt mit der Kleptnerin ist also erstellt und alle Vertragsmodalitäten sind vereinbart. Nun wollen wir an einem bestimmten Tag mit ihr arbeiten. Dafür müssen wir sie zuerst auf die Baustelle bestellen. Übersetzt auf programmieren bedeutet dies, wir müssen die Erweiterung in unsere Session laden. In R sieht der Befehl so aus:

\begin{sphinxVerbatim}[commandchars=\\\{\}]
\PYG{n}{library}\PYG{p}{(}\PYG{n}{malerinMina}\PYG{p}{)}
\end{sphinxVerbatim}

In Python lautet der entsprechende Befehl:

\begin{sphinxVerbatim}[commandchars=\\\{\}]
\PYG{k+kn}{import} \PYG{n+nn}{malerinMina}
\end{sphinxVerbatim}

Erst jetzt können wir mit der Erweiterung arbeiten und die Fachexpertise unserer Malerin nutzen. Eine Expertise unserer Malerin ist es, Wände zu bemalen. Dafür gibt es eine \sphinxstyleemphasis{Function} \sphinxcode{\sphinxupquote{wand\_bemalen()}}. In R kann ich diese \sphinxstyleemphasis{Function} “einfach so” aufrufen:

\begin{sphinxVerbatim}[commandchars=\\\{\}]
\PYG{n}{wand\PYGZus{}bemalen}\PYG{p}{(}\PYG{p}{)}
\end{sphinxVerbatim}

In Python hingegen muss ich die Erweiterung, in der die \sphinxstyleemphasis{Function} enthalten ist, der \sphinxstyleemphasis{Function} mit einem Punkt voranstellen. Das sieht also folgendermassen aus:

\begin{sphinxVerbatim}[commandchars=\\\{\}]
\PYG{n}{malerinMina}\PYG{o}{.}\PYG{n}{wand\PYGZus{}bemalen}\PYG{p}{(}\PYG{p}{)}
\end{sphinxVerbatim}

Das ist zwar umständlicher, aber dafür weniger Fehleranfällig. Angenommen, unser Maurer kann ebenfalls Wände bemalen und hat die entsprechende \sphinxstyleemphasis{Function} \sphinxcode{\sphinxupquote{wand\_bemalen()}} ebenfalls. Dann ist in R nicht klar, welche Erweiterung gemeint ist und das kann zu Missverständnissen führen (vielleicht bemalt der Mauerer die Wände etwas anders als die Malerin). In Python ist im obigen Beispiel unmissverständlich, dass ich \sphinxcode{\sphinxupquote{wand\_bemalen()}} aus dem Modul \sphinxcode{\sphinxupquote{malerinMina}} meine.

\sphinxstylestrong{Das wichtigste in Kürze}
\begin{itemize}
\item {} 
Erweiterungen heissen in Python \sphinxstyleemphasis{Modules}

\item {} 
Vor der erstmaligen Nutzung muss ein \sphinxstyleemphasis{Module} installiert werden

\item {} 
Vor der Verwendung in einer Session muss ein \sphinxstyleemphasis{Module} “geladen” werden. Dies muss für jede Session wiederholt werden!

\item {} 
Ein \sphinxstyleemphasis{Module} wird in Python mit \sphinxcode{\sphinxupquote{import modulename}} geladen

\item {} 
Eine \sphinxstyleemphasis{Function} aus einem \sphinxstyleemphasis{Module} wird folgendermassen aufgerufen: \sphinxcode{\sphinxupquote{modulname.function()}}

\end{itemize}


\section{Modul mit Alias importieren}
\label{\detokenize{01_03_Python_Module:modul-mit-alias-importieren}}
Wenn es uns zu Umständlich ist jedesmal \sphinxcode{\sphinxupquote{malerinMina.wand\_bemalen()}} voll auszuschreiben können wir beim Importieren dem Modul auch einen “Alias” vergeben. Dies kann beispielsweise folgendermassen aussehen:

\begin{sphinxVerbatim}[commandchars=\\\{\}]
\PYG{k+kn}{import} \PYG{n+nn}{malerinMina} \PYG{k}{as} \PYG{n+nn}{mm}
\PYG{n}{mm}\PYG{o}{.}\PYG{n}{wand\PYGZus{}bemalen}\PYG{p}{(}\PYG{p}{)}
\end{sphinxVerbatim}

Dies ist deshalb wichtig, weil sich für viele Module haben sich bestimmte Aliasse eingebürgert haben. Ihr macht sich das Leben leichter, wenn ihr euch an diese Konventionen (welche ihr noch kennenlernen werdet) hält.


\section{Einzelne \sphinxstyleemphasis{Function} importieren}
\label{\detokenize{01_03_Python_Module:einzelne-function-importieren}}
Was man auch noch machen kann, ist eine explizite \sphinxstyleemphasis{Function} aus einem Modul zu laden. Wenn man dies macht, kann man die Funktion ohne vorangestelltes Modul nutzen (genau wie in R). Dies sieht folgendermassen aus:

\begin{sphinxVerbatim}[commandchars=\\\{\}]
\PYG{k+kn}{from} \PYG{n+nn}{malerinMina} \PYG{k+kn}{import} \PYG{n}{wand\PYGZus{}bemalen}\PYG{p}{(}\PYG{p}{)}
\PYG{n}{wand\PYGZus{}bemalen}\PYG{p}{(}\PYG{p}{)}
\end{sphinxVerbatim}


\chapter{Tabellarische Daten}
\label{\detokenize{01_04_Tabellarische_Daten:tabellarische-daten}}\label{\detokenize{01_04_Tabellarische_Daten::doc}}
Schauen wir uns nochmals die \sphinxstyleemphasis{Dictionary} \sphinxcode{\sphinxupquote{people}} an. Diese ist ein Spezialfall einer Dictionary: Jeder Eintrag besteht aus einer Liste von gleich vielen Werten. Diese Dictionary schreit förmlich danach, tabellarisch dargestellt zu werden. Dies wollen wir gewähren und brauchen dafür das Modul \sphinxcode{\sphinxupquote{pandas}}. Importiere das Modul mit dem Alias \sphinxcode{\sphinxupquote{pd}}.

!! ACHTUNG: HIER FEHLT NOCH DIE EINFÜHRUNG INSTALLATION VON MODULEN !!


\section{Übung 1: von einer \sphinxstyleemphasis{Dictionary} zu einer \sphinxstyleemphasis{DataFrame}}
\label{\detokenize{01_04_Tabellarische_Daten:ubung-1-von-einer-dictionary-zu-einer-dataframe}}
Wandle die Dictionary \sphinxcode{\sphinxupquote{people}} in eine DataFrame um: Dazu musst du \sphinxcode{\sphinxupquote{people}} als Argument der Funktion \sphinxcode{\sphinxupquote{DataFrame}} übergeben: \sphinxcode{\sphinxupquote{pd.DataFrame(people)}}. Weise den Output der Variable \sphinxcode{\sphinxupquote{people\_df}} zu.

\sphinxstylestrong{Für R Nutzer}: Auch in R gibt es \sphinxstyleemphasis{Dataframes}, diese sind aber in Base R integriert (brauchen deswegen keine Erweiterung wie \sphinxcode{\sphinxupquote{pandas}})


\section{Übung 2: \sphinxstyleemphasis{DataFrame} in csv umwandeln}
\label{\detokenize{01_04_Tabellarische_Daten:ubung-2-dataframe-in-csv-umwandeln}}
In der Praxis kommen Tabellarische Daten meist als «csv» Dateien daher. Wir können aus unserer eben erstellten DataFrame sehr einfach eine csv Datei erstellen. Führe das mit folgendem Code aus:

\begin{sphinxVerbatim}[commandchars=\\\{\}]
\PYG{n}{people\PYGZus{}df}\PYG{o}{.}\PYG{n}{to\PYGZus{}csv}\PYG{p}{(}\PYG{l+s+s2}{\PYGZdq{}}\PYG{l+s+s2}{people.csv}\PYG{l+s+s2}{\PYGZdq{}}\PYG{p}{)}
\end{sphinxVerbatim}


\section{Übung 3: CSV als \sphinxstyleemphasis{DataFrame} importieren}
\label{\detokenize{01_04_Tabellarische_Daten:ubung-3-csv-als-dataframe-importieren}}
Genau so einfach ist es eine csv zu importieren. Lade dazu die Datei “zeckenstiche.csv” von Moodle herunter und speichere es im aktuellen Arbeitsverzeichnis ab. Importiere mit folgendem Code die Datei “zeckenstiche.csv”. Schau dir \sphinxcode{\sphinxupquote{zeckenstiche}} nach dem importieren an.

\begin{sphinxVerbatim}[commandchars=\\\{\}]
\PYG{n}{zeckenstiche} \PYG{o}{=} \PYG{n}{pd}\PYG{o}{.}\PYG{n}{read\PYGZus{}csv}\PYG{p}{(}\PYG{l+s+s2}{\PYGZdq{}}\PYG{l+s+s2}{zeckenstiche.csv}\PYG{l+s+s2}{\PYGZdq{}}\PYG{p}{)}
\end{sphinxVerbatim}

\sphinxstylestrong{Hinweis}: Dieser Code funktioniert nur, wenn “zeckenstiche.csv” im aktuellen Arbeitsverzeichnis (\sphinxstyleemphasis{Current Working Directory}) abgespeichert wird. Wenn du nicht sicher bist, wo dein aktuelles Arbeitsverzeichnis liegt, kannst du das Modul \sphinxcode{\sphinxupquote{os}} laden und dies mit der Funktion \sphinxcode{\sphinxupquote{os.getcwd()}} (get \sphinxstylestrong{c}urrent\sphinxstylestrong{w}orking\sphinxstylestrong{d}irectory) herausfinden.

\begin{sphinxVerbatim}[commandchars=\\\{\}]
\PYG{k+kn}{import} \PYG{n+nn}{os}

\PYG{n}{os}\PYG{o}{.}\PYG{n}{getcwd}\PYG{p}{(}\PYG{p}{)}
\end{sphinxVerbatim}

\begin{sphinxVerbatim}[commandchars=\\\{\}]
\PYGZsq{}/home/rata/codingingis\PYGZus{}book\PYGZsq{}
\end{sphinxVerbatim}


\section{Übung 4: Koordinaten räumlich darstellen}
\label{\detokenize{01_04_Tabellarische_Daten:ubung-4-koordinaten-raumlich-darstellen}}
Die \sphinxstyleemphasis{DataFrame} \sphinxcode{\sphinxupquote{zeckenstiche}} beinhaltet x und y Koordinaten für jeden Unfall in den gleichnamigen Spalten. Wir können die Stiche mit einem Scatterplot räumlich visualisieren. Führe dazu folgenden Code aus:

\begin{sphinxVerbatim}[commandchars=\\\{\}]
\PYG{n}{zeckenstiche}\PYG{o}{.}\PYG{n}{plot}\PYG{o}{.}\PYG{n}{scatter}\PYG{p}{(}\PYG{l+s+s2}{\PYGZdq{}}\PYG{l+s+s2}{x}\PYG{l+s+s2}{\PYGZdq{}}\PYG{p}{,}\PYG{l+s+s2}{\PYGZdq{}}\PYG{l+s+s2}{y}\PYG{l+s+s2}{\PYGZdq{}}\PYG{p}{)}
\end{sphinxVerbatim}

\begin{sphinxVerbatim}[commandchars=\\\{\}]
\PYGZlt{}AxesSubplot:xlabel=\PYGZsq{}x\PYGZsq{}, ylabel=\PYGZsq{}y\PYGZsq{}\PYGZgt{}
\end{sphinxVerbatim}

\noindent\sphinxincludegraphics{{01_04_Tabellarische_Daten_14_1}.png}


\section{Übung 5: Einzelne Spalte selektieren}
\label{\detokenize{01_04_Tabellarische_Daten:ubung-5-einzelne-spalte-selektieren}}
Um eine einzelne Spalte zu selektieren (z.B. die Spalte “ID”) kann man gleich vorgehen wie bei der Selektion eines Eintrage in einer \sphinxstyleemphasis{Dictionary}. Probiere es aus.

\begin{sphinxVerbatim}[commandchars=\\\{\}]
\PYG{n}{zeckenstiche}\PYG{p}{[}\PYG{l+s+s2}{\PYGZdq{}}\PYG{l+s+s2}{ID}\PYG{l+s+s2}{\PYGZdq{}}\PYG{p}{]}
\end{sphinxVerbatim}

\begin{sphinxVerbatim}[commandchars=\\\{\}]
0          0
1          1
2          2
3          3
4          4
        ... 
1071    1071
1072    1072
1073    1073
1074    1074
1075    1075
Name: ID, Length: 1076, dtype: int64
\end{sphinxVerbatim}


\section{Übung 6: Neue Spalte erstellen}
\label{\detokenize{01_04_Tabellarische_Daten:ubung-6-neue-spalte-erstellen}}
Auch das Erstellen einer neuen Spalte ist identisch mit der Erstellung eines neuen \sphinxstyleemphasis{Dictionary} Eintrags. Folgende Zeile erstellt eine neue Spalte “Stichtyp” mit dem gleichen Wert in allen Zeilen (“Zecke”). Erstelle ebenfalls eine solche Spalte.

\begin{sphinxVerbatim}[commandchars=\\\{\}]
\PYG{n}{zeckenstiche}\PYG{p}{[}\PYG{l+s+s2}{\PYGZdq{}}\PYG{l+s+s2}{Stichtyp}\PYG{l+s+s2}{\PYGZdq{}}\PYG{p}{]} \PYG{o}{=} \PYG{l+s+s2}{\PYGZdq{}}\PYG{l+s+s2}{Zecke}\PYG{l+s+s2}{\PYGZdq{}}

\PYG{n}{zeckenstiche}
\end{sphinxVerbatim}

\begin{sphinxVerbatim}[commandchars=\\\{\}]
        ID    accuracy          x          y Stichtyp
0        0   65.617154  2678971.0  1240824.0    Zecke
1        1  257.492052  2679837.0  1240858.0    Zecke
2        2  163.533834  2687539.0  1240881.0    Zecke
3        3  185.000000  2674848.0  1240913.0    Zecke
4        4  228.215231  2681771.0  1240922.0    Zecke
...    ...         ...        ...        ...      ...
1071  1071  109.531946  2678005.0  1257344.0    Zecke
1072  1072  100.489274  2678005.0  1257347.0    Zecke
1073  1073  301.748529  2689893.0  1257351.0    Zecke
1074  1074  301.748542  2690668.0  1257369.0    Zecke
1075  1075  226.000000  2682852.0  1257531.0    Zecke

[1076 rows x 5 columns]
\end{sphinxVerbatim}


\chapter{Einleitung zur Übung Coding in GIS II}
\label{\detokenize{02_01_Einleitung:einleitung-zur-ubung-coding-in-gis-ii}}\label{\detokenize{02_01_Einleitung::doc}}
Letzte Woche habt ihr einen weiten Bogen gespannt: Von einfachen (primitiven) Datentypen bis hin zu Geodaten und deren Visualisierung. In dieser und der nächsten Übung (Coding in GIS II und III) wollen wir uns weiter mit den Zeckenstichdaten befassen. Wir werden im Wesentlichen ein Teil der Übung aus «Datenqualität und Unsicherheit» in Python rekonstruieren.
In der Übung geht es um folgendes: Wir wissen das die Lagegenauigkeit der Zeckenstichmeldungen mit einer gewissen Unsicherheit behaftet sind. Um die Frage “Welcher Anteil der der Zeckenstiche befinden sich im Wald?” unter Berücksichtigung dieser Unsicherheit beantworten zu können, führen wir eine (Monte Carlo) Simulation durch.
In dieser Simulation verändern wir die Position der Zeckenstichmeldungen zufällig und berechnen den Anteil der Zeckenstiche im Wald. Das zufällige Verschieben und berechnen wiederholen wir beliebig lange und bekommen für jede Wiederholung einen leicht unterschiedlichen Prozentwert. Die Verteilung dieser Prozentwerte ist die Antwort auf die ursprüngliche Frage (“Welcher Anteil…”) unter Berücksichtigung der Unsicherheit.

Um eine solche, etwas komplexere Aufgabe lösen zu können müssen wir sie in einfachere Einzelschritte aufteilen. Diese bearbeiten wir in dieser und der kommenden Woche:
\begin{itemize}
\item {} 
Schritt 1: Einen Einzelpunkt zufällig verschieben

\item {} 
Schritt 2: Alle Punkte einer GeoDataFrame zufällig verschieben (1 «Run»)

\item {} 
Schritt 3: Alle Punkte einer GeoDataFrame mehrfach zufällig verschieben (z.B. 50 «Runs»)

\item {} 
Schritt 4: Anteil der Punkte im Wald pro «Run» ermitteln

\item {} 
Schritt 5: Verteilung dieser Mittelwerte visualisieren

\end{itemize}


\chapter{Functions}
\label{\detokenize{02_02_Functions:functions}}\label{\detokenize{02_02_Functions::doc}}
Bevor wir uns in die Simulation stürzen müssen wir noch lernen eigene \sphinxstyleemphasis{Functions} zu schreiben: Dies ist auch nicht weiter schwierig: Eine \sphinxstyleemphasis{Function} wird mit \sphinxcode{\sphinxupquote{def}} eingeleitet, braucht einen Namen, einen Input und einen Output.

Wenn wir zum Beispiel eine Function erstellen wollen die uns grüsst, so geht dies folgendermassen:

\begin{sphinxVerbatim}[commandchars=\\\{\}]
\PYG{k}{def} \PYG{n+nf}{sag\PYGZus{}hallo}\PYG{p}{(}\PYG{p}{)}\PYG{p}{:}
    \PYG{k}{return}\PYG{p}{(}\PYG{l+s+s2}{\PYGZdq{}}\PYG{l+s+s2}{Hallo!}\PYG{l+s+s2}{\PYGZdq{}}\PYG{p}{)}
\end{sphinxVerbatim}
\begin{itemize}
\item {} 
Mit \sphinxcode{\sphinxupquote{def}} sagen wir: “Jetzt definiere ich eine Function”.

\item {} 
Danach kommt der Name der \sphinxstyleemphasis{Function}, in unserem Fall \sphinxcode{\sphinxupquote{sag\_hallo}} (mit diesem Namen können wir die \sphinxstyleemphasis{Function} später wieder abrufen).

\item {} 
Als drittes kommen die runden Klammern, wo wir bei Bedarf Inputvariablen (sogenannte Parameter) festlegen können. In diesem ersten Beispiel habe ich keine Parameter festgelegt

\item {} 
Nach der Klammer kommt ein Doppelpunkt was bedeutet: “jetzt wird gleich definiert, was die Funktion tun soll”

\item {} 
Auf einer neuen Zeile wird eingerückt festgelegt, was die Function eben tun soll. Meist sind hier ein paar Zeilen Code vorhanden

\item {} 
Die letzte eingerückte Zeile (in unserem Fall ist das die einzige Zeile) gibt mit return an, was die \sphinxstyleemphasis{Function} zurück geben soll (der Output). In unserem Fall soll sie “Hallo!”  zurück geben.

\item {} 
Voila, das war’s schon! Jetzt können wir diese Function schon nutzen:

\end{itemize}

\begin{sphinxVerbatim}[commandchars=\\\{\}]
\PYG{n}{sag\PYGZus{}hallo}\PYG{p}{(}\PYG{p}{)}
\end{sphinxVerbatim}

\begin{sphinxVerbatim}[commandchars=\\\{\}]
\PYGZsq{}Hallo!\PYGZsq{}
\end{sphinxVerbatim}

Diese \sphinxstyleemphasis{Function} ohne Input ist wenig nützlich. Meist wollen wir der \sphinxstyleemphasis{Function} etwas \sphinxhyphen{} einen Input \sphinxhyphen{} übergeben können. Um eine \sphinxstyleemphasis{Function} zu erstellen die ein Argument annimmt geht man folgendermassen vor:

\begin{sphinxVerbatim}[commandchars=\\\{\}]
\PYG{k}{def} \PYG{n+nf}{sag\PYGZus{}hallo}\PYG{p}{(}\PYG{n}{vorname}\PYG{p}{)}\PYG{p}{:}
    \PYG{k}{return}\PYG{p}{(}\PYG{l+s+s2}{\PYGZdq{}}\PYG{l+s+s2}{Hallo }\PYG{l+s+s2}{\PYGZdq{}}\PYG{o}{+}\PYG{n}{vorname}\PYG{o}{+}\PYG{l+s+s2}{\PYGZdq{}}\PYG{l+s+s2}{!}\PYG{l+s+s2}{\PYGZdq{}}\PYG{p}{)}
\end{sphinxVerbatim}

Nun können wir der Function einen Paramter übergeben, damit wir persönlich gegrüsst werden. In folgendem Beispiel ist \sphinxcode{\sphinxupquote{vorname}} ein Parameter, “Charles” ist sein Argument.

\begin{sphinxVerbatim}[commandchars=\\\{\}]
\PYG{n}{sag\PYGZus{}hallo}\PYG{p}{(}\PYG{l+s+s2}{\PYGZdq{}}\PYG{l+s+s2}{Charles}\PYG{l+s+s2}{\PYGZdq{}}\PYG{p}{)}
\end{sphinxVerbatim}

\begin{sphinxVerbatim}[commandchars=\\\{\}]
\PYGZsq{}Hallo Charles!\PYGZsq{}
\end{sphinxVerbatim}

Den Output können wir wie gewohnt einer neuen Variabel zuweisen:

\begin{sphinxVerbatim}[commandchars=\\\{\}]
\PYG{n}{persoenlicher\PYGZus{}gruss} \PYG{o}{=} \PYG{n}{sag\PYGZus{}hallo}\PYG{p}{(}\PYG{l+s+s2}{\PYGZdq{}}\PYG{l+s+s2}{Charles}\PYG{l+s+s2}{\PYGZdq{}}\PYG{p}{)}
\PYG{n}{persoenlicher\PYGZus{}gruss}
\end{sphinxVerbatim}

\begin{sphinxVerbatim}[commandchars=\\\{\}]
\PYGZsq{}Hallo Charles!\PYGZsq{}
\end{sphinxVerbatim}







\renewcommand{\indexname}{Index}
\printindex
\end{document}